\documentclass[11pt]{article}
\def\documentlanguage{slovak}
\usepackage{framed}
\usepackage{fancyvrb}
\usepackage{color}
\usepackage{listings}
\usepackage{ifpdf}
\usepackage{ifxetex}
\usepackage{graphicx}
\usepackage{longtable}
\ifpdf %pdftex
\usepackage[utf8]{inputenc}
\usepackage[T1]{fontenc}
\usepackage[all,pdf,2cell]{xy}\UseAllTwocells\SilentMatrices
\usepackage[\documentlanguage]{babel}
\fi
\ifxetex %xetex
\usepackage[all,pdf,2cell]{xy}\UseAllTwocells\SilentMatrices
\usepackage{polyglossia}
\begingroup\edef\x{\endgroup\noexpand\setdefaultlanguage{\documentlanguage}}\x%
\fi
\usepackage{amsthm}
\usepackage{amsmath}
\usepackage{amsfonts}
\newtheorem{theorem}{Theorem}[section]
\newtheorem{lemma}[theorem]{Lemma}
\newtheorem{proposition}[theorem]{Proposition}
\theoremstyle{definition}
\newtheorem{definition}[theorem]{Definition}
\newtheorem{example}[theorem]{Príklad}
\newcommand{\newcategory}[1]{\expandafter\newcommand\csname #1\endcsname{\mathbf{#1}}}

\makeatletter
\def\PY@reset{\let\PY@it=\relax \let\PY@bf=\relax%
    \let\PY@ul=\relax \let\PY@tc=\relax%
    \let\PY@bc=\relax \let\PY@ff=\relax}
\def\PY@tok#1{\csname PY@tok@#1\endcsname}
\def\PY@toks#1+{\ifx\relax#1\empty\else%
    \PY@tok{#1}\expandafter\PY@toks\fi}
\def\PY@do#1{\PY@bc{\PY@tc{\PY@ul{%
    \PY@it{\PY@bf{\PY@ff{#1}}}}}}}
\def\PY#1#2{\PY@reset\PY@toks#1+\relax+\PY@do{#2}}

\expandafter\def\csname PY@tok@w\endcsname{\def\PY@tc##1{\textcolor[rgb]{0.73,0.73,0.73}{##1}}}
\expandafter\def\csname PY@tok@c\endcsname{\let\PY@it=\textit\def\PY@tc##1{\textcolor[rgb]{0.25,0.50,0.50}{##1}}}
\expandafter\def\csname PY@tok@cp\endcsname{\def\PY@tc##1{\textcolor[rgb]{0.74,0.48,0.00}{##1}}}
\expandafter\def\csname PY@tok@k\endcsname{\let\PY@bf=\textbf\def\PY@tc##1{\textcolor[rgb]{0.00,0.50,0.00}{##1}}}
\expandafter\def\csname PY@tok@kp\endcsname{\def\PY@tc##1{\textcolor[rgb]{0.00,0.50,0.00}{##1}}}
\expandafter\def\csname PY@tok@kt\endcsname{\def\PY@tc##1{\textcolor[rgb]{0.69,0.00,0.25}{##1}}}
\expandafter\def\csname PY@tok@o\endcsname{\def\PY@tc##1{\textcolor[rgb]{0.40,0.40,0.40}{##1}}}
\expandafter\def\csname PY@tok@ow\endcsname{\let\PY@bf=\textbf\def\PY@tc##1{\textcolor[rgb]{0.67,0.13,1.00}{##1}}}
\expandafter\def\csname PY@tok@nb\endcsname{\def\PY@tc##1{\textcolor[rgb]{0.00,0.50,0.00}{##1}}}
\expandafter\def\csname PY@tok@nf\endcsname{\def\PY@tc##1{\textcolor[rgb]{0.00,0.00,1.00}{##1}}}
\expandafter\def\csname PY@tok@nc\endcsname{\let\PY@bf=\textbf\def\PY@tc##1{\textcolor[rgb]{0.00,0.00,1.00}{##1}}}
\expandafter\def\csname PY@tok@nn\endcsname{\let\PY@bf=\textbf\def\PY@tc##1{\textcolor[rgb]{0.00,0.00,1.00}{##1}}}
\expandafter\def\csname PY@tok@ne\endcsname{\let\PY@bf=\textbf\def\PY@tc##1{\textcolor[rgb]{0.82,0.25,0.23}{##1}}}
\expandafter\def\csname PY@tok@nv\endcsname{\def\PY@tc##1{\textcolor[rgb]{0.10,0.09,0.49}{##1}}}
\expandafter\def\csname PY@tok@no\endcsname{\def\PY@tc##1{\textcolor[rgb]{0.53,0.00,0.00}{##1}}}
\expandafter\def\csname PY@tok@nl\endcsname{\def\PY@tc##1{\textcolor[rgb]{0.63,0.63,0.00}{##1}}}
\expandafter\def\csname PY@tok@ni\endcsname{\let\PY@bf=\textbf\def\PY@tc##1{\textcolor[rgb]{0.60,0.60,0.60}{##1}}}
\expandafter\def\csname PY@tok@na\endcsname{\def\PY@tc##1{\textcolor[rgb]{0.49,0.56,0.16}{##1}}}
\expandafter\def\csname PY@tok@nt\endcsname{\let\PY@bf=\textbf\def\PY@tc##1{\textcolor[rgb]{0.00,0.50,0.00}{##1}}}
\expandafter\def\csname PY@tok@nd\endcsname{\def\PY@tc##1{\textcolor[rgb]{0.67,0.13,1.00}{##1}}}
\expandafter\def\csname PY@tok@s\endcsname{\def\PY@tc##1{\textcolor[rgb]{0.73,0.13,0.13}{##1}}}
\expandafter\def\csname PY@tok@sd\endcsname{\let\PY@it=\textit\def\PY@tc##1{\textcolor[rgb]{0.73,0.13,0.13}{##1}}}
\expandafter\def\csname PY@tok@si\endcsname{\let\PY@bf=\textbf\def\PY@tc##1{\textcolor[rgb]{0.73,0.40,0.53}{##1}}}
\expandafter\def\csname PY@tok@se\endcsname{\let\PY@bf=\textbf\def\PY@tc##1{\textcolor[rgb]{0.73,0.40,0.13}{##1}}}
\expandafter\def\csname PY@tok@sr\endcsname{\def\PY@tc##1{\textcolor[rgb]{0.73,0.40,0.53}{##1}}}
\expandafter\def\csname PY@tok@ss\endcsname{\def\PY@tc##1{\textcolor[rgb]{0.10,0.09,0.49}{##1}}}
\expandafter\def\csname PY@tok@sx\endcsname{\def\PY@tc##1{\textcolor[rgb]{0.00,0.50,0.00}{##1}}}
\expandafter\def\csname PY@tok@m\endcsname{\def\PY@tc##1{\textcolor[rgb]{0.40,0.40,0.40}{##1}}}
\expandafter\def\csname PY@tok@gh\endcsname{\let\PY@bf=\textbf\def\PY@tc##1{\textcolor[rgb]{0.00,0.00,0.50}{##1}}}
\expandafter\def\csname PY@tok@gu\endcsname{\let\PY@bf=\textbf\def\PY@tc##1{\textcolor[rgb]{0.50,0.00,0.50}{##1}}}
\expandafter\def\csname PY@tok@gd\endcsname{\def\PY@tc##1{\textcolor[rgb]{0.63,0.00,0.00}{##1}}}
\expandafter\def\csname PY@tok@gi\endcsname{\def\PY@tc##1{\textcolor[rgb]{0.00,0.63,0.00}{##1}}}
\expandafter\def\csname PY@tok@gr\endcsname{\def\PY@tc##1{\textcolor[rgb]{1.00,0.00,0.00}{##1}}}
\expandafter\def\csname PY@tok@ge\endcsname{\let\PY@it=\textit}
\expandafter\def\csname PY@tok@gs\endcsname{\let\PY@bf=\textbf}
\expandafter\def\csname PY@tok@gp\endcsname{\let\PY@bf=\textbf\def\PY@tc##1{\textcolor[rgb]{0.00,0.00,0.50}{##1}}}
\expandafter\def\csname PY@tok@go\endcsname{\def\PY@tc##1{\textcolor[rgb]{0.53,0.53,0.53}{##1}}}
\expandafter\def\csname PY@tok@gt\endcsname{\def\PY@tc##1{\textcolor[rgb]{0.00,0.27,0.87}{##1}}}
\expandafter\def\csname PY@tok@err\endcsname{\def\PY@bc##1{\setlength{\fboxsep}{0pt}\fcolorbox[rgb]{1.00,0.00,0.00}{1,1,1}{\strut ##1}}}
\expandafter\def\csname PY@tok@kc\endcsname{\let\PY@bf=\textbf\def\PY@tc##1{\textcolor[rgb]{0.00,0.50,0.00}{##1}}}
\expandafter\def\csname PY@tok@kd\endcsname{\let\PY@bf=\textbf\def\PY@tc##1{\textcolor[rgb]{0.00,0.50,0.00}{##1}}}
\expandafter\def\csname PY@tok@kn\endcsname{\let\PY@bf=\textbf\def\PY@tc##1{\textcolor[rgb]{0.00,0.50,0.00}{##1}}}
\expandafter\def\csname PY@tok@kr\endcsname{\let\PY@bf=\textbf\def\PY@tc##1{\textcolor[rgb]{0.00,0.50,0.00}{##1}}}
\expandafter\def\csname PY@tok@bp\endcsname{\def\PY@tc##1{\textcolor[rgb]{0.00,0.50,0.00}{##1}}}
\expandafter\def\csname PY@tok@fm\endcsname{\def\PY@tc##1{\textcolor[rgb]{0.00,0.00,1.00}{##1}}}
\expandafter\def\csname PY@tok@vc\endcsname{\def\PY@tc##1{\textcolor[rgb]{0.10,0.09,0.49}{##1}}}
\expandafter\def\csname PY@tok@vg\endcsname{\def\PY@tc##1{\textcolor[rgb]{0.10,0.09,0.49}{##1}}}
\expandafter\def\csname PY@tok@vi\endcsname{\def\PY@tc##1{\textcolor[rgb]{0.10,0.09,0.49}{##1}}}
\expandafter\def\csname PY@tok@vm\endcsname{\def\PY@tc##1{\textcolor[rgb]{0.10,0.09,0.49}{##1}}}
\expandafter\def\csname PY@tok@sa\endcsname{\def\PY@tc##1{\textcolor[rgb]{0.73,0.13,0.13}{##1}}}
\expandafter\def\csname PY@tok@sb\endcsname{\def\PY@tc##1{\textcolor[rgb]{0.73,0.13,0.13}{##1}}}
\expandafter\def\csname PY@tok@sc\endcsname{\def\PY@tc##1{\textcolor[rgb]{0.73,0.13,0.13}{##1}}}
\expandafter\def\csname PY@tok@dl\endcsname{\def\PY@tc##1{\textcolor[rgb]{0.73,0.13,0.13}{##1}}}
\expandafter\def\csname PY@tok@s2\endcsname{\def\PY@tc##1{\textcolor[rgb]{0.73,0.13,0.13}{##1}}}
\expandafter\def\csname PY@tok@sh\endcsname{\def\PY@tc##1{\textcolor[rgb]{0.73,0.13,0.13}{##1}}}
\expandafter\def\csname PY@tok@s1\endcsname{\def\PY@tc##1{\textcolor[rgb]{0.73,0.13,0.13}{##1}}}
\expandafter\def\csname PY@tok@mb\endcsname{\def\PY@tc##1{\textcolor[rgb]{0.40,0.40,0.40}{##1}}}
\expandafter\def\csname PY@tok@mf\endcsname{\def\PY@tc##1{\textcolor[rgb]{0.40,0.40,0.40}{##1}}}
\expandafter\def\csname PY@tok@mh\endcsname{\def\PY@tc##1{\textcolor[rgb]{0.40,0.40,0.40}{##1}}}
\expandafter\def\csname PY@tok@mi\endcsname{\def\PY@tc##1{\textcolor[rgb]{0.40,0.40,0.40}{##1}}}
\expandafter\def\csname PY@tok@il\endcsname{\def\PY@tc##1{\textcolor[rgb]{0.40,0.40,0.40}{##1}}}
\expandafter\def\csname PY@tok@mo\endcsname{\def\PY@tc##1{\textcolor[rgb]{0.40,0.40,0.40}{##1}}}
\expandafter\def\csname PY@tok@ch\endcsname{\let\PY@it=\textit\def\PY@tc##1{\textcolor[rgb]{0.25,0.50,0.50}{##1}}}
\expandafter\def\csname PY@tok@cm\endcsname{\let\PY@it=\textit\def\PY@tc##1{\textcolor[rgb]{0.25,0.50,0.50}{##1}}}
\expandafter\def\csname PY@tok@cpf\endcsname{\let\PY@it=\textit\def\PY@tc##1{\textcolor[rgb]{0.25,0.50,0.50}{##1}}}
\expandafter\def\csname PY@tok@c1\endcsname{\let\PY@it=\textit\def\PY@tc##1{\textcolor[rgb]{0.25,0.50,0.50}{##1}}}
\expandafter\def\csname PY@tok@cs\endcsname{\let\PY@it=\textit\def\PY@tc##1{\textcolor[rgb]{0.25,0.50,0.50}{##1}}}

\def\PYZbs{\char`\\}
\def\PYZus{\char`\_}
\def\PYZob{\char`\{}
\def\PYZcb{\char`\}}
\def\PYZca{\char`\^}
\def\PYZam{\char`\&}
\def\PYZlt{\char`\<}
\def\PYZgt{\char`\>}
\def\PYZsh{\char`\#}
\def\PYZpc{\char`\%}
\def\PYZdl{\char`\$}
\def\PYZhy{\char`\-}
\def\PYZsq{\char`\'}
\def\PYZdq{\char`\"}
\def\PYZti{\char`\~}
% for compatibility with earlier versions
\def\PYZat{@}
\def\PYZlb{[}
\def\PYZrb{]}
\makeatother


\begin{document}
\title{Projekt na získanie zápočtu z predmetu Počítačové siete\\LS 2019/20}
\maketitle
\section{Získanie zápočtu}
\subsection{Formálne aspekty}

\begin{itemize}
\item Odovzdávať sa bude prostredníctvom githubu, sprístupnením repozitára menom {\tt astack} užívateľovi {\tt gjenca}
\item Repozitár bude obsahovať iba jeden súbor so zdrojovým kódom, nič iné.
\item Termín odovzdania projektu je 15.6.2020 polnoc.
\item Po tomto termíne budú projekty vyhodnotené do troch dní, známka bude zapísaná do AIS a študentom budú oznámené
požadované zmeny pre zlepšenie známky.
\item V prípade, že niekto bude mať záujem o zlepšenie známky, bude vypísaný ďalší termín dokedy budú môcť záujemcovia
odovzdať druhú verziu projektu.
\end{itemize}

\subsection{Kritériá hodnotenia}
\begin{itemize}
\item Posudzovať sa bude korektnosť implementácie v zmysle špecifikácie protokolu.
\item Posudzovať sa bude aj zdrojový kód: pokiaľ niekto odovzdá príliš škaredý kód, dostane horšiu známku.
\item Používajte, (adekvátne účelu) prostriedky Pythonu: funkcie, slovníky, triedy, zabudované dátové typy.
\end{itemize}

\section{Aritmetický zásobník}
{\em Aritmetický zásobník} je dátová štruktúra, ktorá umožňuje vykonávať výpočty. Pre zjednodušenie
situácie budeme uvažovať iba o prirodzených číslach a iba o sčítaní a násobení.
\begin{example}
Ukážme si, ako prebieha pomocou aritmetického zásobníka výpočet hodnoty 
jednoduchého výrazu $7.10+3.5$.
\begin{enumerate}
\item Na začiatku máme prázdny zásobník.
\begin{center}
\includegraphics{stack_empty}
\end{center}
\item Vložíme do zásobníka $7$
\begin{center}
\includegraphics{stack_7}
\end{center}
\item Vložíme do zásobníka $10$
\begin{center}
\includegraphics{stack_7_10}
\end{center}
\item Vyberieme dve čísla na vrchu zásobníka, vynásobíme ich, výsledok vložíme na vrch zásobníka.
\begin{center}
\includegraphics{stack_70}
\end{center}
\item Vložíme do zásobníka čísla $3$ a $5$.
\begin{center}
\includegraphics{stack_70_3_5}
\end{center}
\item Vyberieme dve čísla na vrchu zásobníka, vynásobíme ich, výsledok vložíme na vrch zásobníka.
\begin{center}
\includegraphics{stack_70_15}
\end{center}
\item Vyberieme dve čísla na vrchu zásobníka, sčítame ich, výsledok vložíme na vrch zásobníka.
\begin{center}
\includegraphics{stack_85}
\end{center}
\end{enumerate}
Výsledok je na vrchu zásobníka.
\end{example}
Vidíme, že aritmetický zásobník umožňuje v jednom kroku tieto vykonať tieto veci:
\begin{enumerate}
\item Vložiť jedno alebo viac prirodzených čísel na vrch zásobníka.
\item Vybrať dve čísla na vrchu zásobníka, vynásobiť ich, výsledok vložiť na vrch zásobníka.
\item Vybrať dve čísla na vrchu zásobníka, sčítať ich, výsledok vložiť na vrch zásobníka.
\end{enumerate}
Navyše, samozrejme, musí umožňovať zistiť aké číslo je na vrchu zásobníka, aby sme vedeli určiť výsledok výpočtu. Dobre
by bolo aj to, aby bolo možné zásobník vyprázdniť, aby sme vedeli začať ďalší výpočet.

\section{Protokol pre aritmetický zásobník}

Chceme teraz navrhnúť protokol, ktorý umožní klientovi používať aritmetický zásobník, ktorý je uložený na serveri.
Najskôr špecifikujeme základné pravidlá.

\subsection{Základné pravidlá}

\begin{itemize}
\item Protokol bude textový, riadkovo orientovaný.
\item Bude postavený nad TCP.
\item Po pripojení klienta vytvorí server nový zásobník, špeciálne pre klienta.
\item Klient bude serveru posielať \emph{požiadavky} (anglicky \emph{request}) a server po každej požiadavke pošle
naspäť \emph{odpoveď} (anglicky \emph{response}).
\item Každá požiadavka a každá odpoveď budú pozostávať z jedného alebo viacerých neprázdnych riadkov, ktoré budú
nasledované prázdnym riadkom.
\item Požiadavka bude vždy obsahovať v prvom riadku \emph{metódu}, t.j. jedno slovo z množiny slov 
{\tt PUSH, MULTIPLY, ADD, PEEK, ZAP}.
\item Nasledovať bude obsah požiadavky: nula alebo viac neprázdnych riadkov.
\item Potom bude vždy práve jeden prázdny riadok.
\item Odpoveď bude mať v prvom riadku \emph{stav} (anglicky status).
\item Stav obsahuje ako prvé slovo trojciferné číslo, nasledované popisom. Príklad stavu:
\begin{flushleft}
{\tt 202 Not a number}
\end{flushleft}
\item Nasledovať bude nula alebo viac riadkov s obsahom odpovede.
\item Potom bude vždy práve jeden prázdny riadok.
\item Spojenie za normálnych okolností ukončí klient.
\end{itemize}

\subsection{Metódy}
Teraz ideme špecifikovať po jednotlivých metódach typy rôznych požiadaviek, možných odpovedí a tomu
zodpovedajúcich efektov (efekt je v tomto kontexte {\em zmena obsahu zásobníka} a prípadne obsahov odpovedí.
Ak nie je v tabuľke v poslednom stĺpci uvedený žiaden obsah odpovede, obsah odpovede
je vtedy prázdny.
\subsubsection{\tt PUSH}
Obsah požiadavky je jedno a viac prirodzených čísel, každé na osobitnom riadku.
\begin{longtable}{|p{12em}|p{12em}|p{15em}|}
\hline
	\textbf{Situácia}
	&
	\textbf{Stav}
	& 
	\textbf{Efekt, prípadný obsah odpovede}\\
\endhead
\hline
	Žiadne čísla neboli poslané, iba prázdny riadok po {\tt PUSH}.
	&	
	201 Request content empty
	&
	Žiaden efekt.\\
\hline
	Na niektorom riadku obsahu požiadavky sa nachádza niečo iné ako prirodzené číslo.
	&	
	202 Not a number
	&
	Žiaden, t.j. požiadavka ako celok je ignorovaná, včítane prípadne správnych čísel.\\
\hline
	Všetko v poriadku. 
	&
	100 OK
	&
	Uloží čísla do zásobníka, prvé naspodku.\\
\hline
\end{longtable}
\subsubsection{\tt ADD}
Obsah požiadavky je prázdny.
\begin{longtable}{|p{12em}|p{12em}|p{15em}|}
\hline
	\textbf{Situácia}
	&
	\textbf{Odpoveď}
	& 
	\textbf{Efekt}\\
\endhead
\hline
	Neprázdny obsah požiadavky.
	&	
	204 Request content nonempty
	&
	Žiaden efekt.\\
\hline
	Zásobník má iba nula alebo jeden prvok.
	&	
	203 Stack too short
	&
	Žiaden efekt.\\
\hline
	Všetko v poriadku. 
	&
	100 OK
	&
	Efekt: vyberie dve čísla z vrchu zásobníka, sčíta ich a výsledok položí na vrch zásobníka.\\
\hline
\end{longtable}
\subsubsection{\tt MULTIPLY}
Obsah požiadavky je prázdny.
\begin{longtable}{|p{12em}|p{12em}|p{15em}|}
\hline
	\textbf{Situácia}
	&
	\textbf{Odpoveď}
	& 
	\textbf{Efekt}\\
\endhead
\hline
	Neprázdny obsah požiadavky.
	&	
	204 Request content nonempty
	&
	Žiaden efekt.\\
\hline
	Zásobník má iba nula alebo jeden prvok.
	&	
	203 Stack too short
	&
	Žiaden efekt.\\
\hline
	Všetko v poriadku. 
	&
	100 OK
	&
	Efekt: vyberie dve čísla z vrchu zásobníka, vynásobí ich ich a výsledok položí na vrch zásobníka.\\
\hline
\end{longtable}
\newpage
\subsubsection{\tt PEEK}
Obsah požiadavky je prázdny.
\begin{longtable}{|p{12em}|p{12em}|p{15em}|}
\hline
	\textbf{Situácia}
	&
	\textbf{Odpoveď}
	& 
	\textbf{Efekt}\\
\endhead
\hline
	Neprázdny obsah požiadavky.
	&	
	204 Request content nonempty
	&
	Žiaden efekt.\\
\hline
	Zásobník je prázdny.
	&	
	205 Stack empty.
	&
	Žiaden efekt.\\
\hline
	Všetko v poriadku. 
	&
	100 OK
	&
	Žiaden efekt.
	Číslo na vrchu zásobníka pošle v obsahu odpovede.\\
\hline
\end{longtable}
\subsubsection{\tt ZAP}
Obsah požiadavky je prázdny.
\begin{longtable}{|p{12em}|p{12em}|p{15em}|}
\hline
	\textbf{Situácia}
	&
	\textbf{Odpoveď}
	& 
	\textbf{Efekt}\\
\endhead
\hline
	Neprázdny obsah požiadavky.
	&	
	204 Nonempty request content.
	&
	Žiaden efekt.\\
\hline
	Všetko v poriadku. 
	&
	100 OK
	&
	Efekt: vyprázdni zásobník.\\
\hline
\end{longtable}
\subsubsection{Zlá požiadavka}
Ak klient pošle na začiatku požiadavky čokoľvek iné, ako je
riadok obsahujúci známu metódu, server vráti {\tt 301 Bad request} a uzavrie spojenie.

\subsection{Príklad komunikácie klient-server}
{\tt C->S} sú označené riadky zasielané klientom serveru, {\tt S->C} naopak. Výraz je $30.20+10$.
\begin{framed}
\begin{Verbatim}[commandchars=\\\{\}]
C\PYZhy{}\PYZgt{}S PEEK
C\PYZhy{}\PYZgt{}S
S\PYZhy{}\PYZgt{}C \PY{l+m}{205} Stack empty
S\PYZhy{}\PYZgt{}C
C\PYZhy{}\PYZgt{}S PUSH  
C\PYZhy{}\PYZgt{}S \PY{l+m}{10}
C\PYZhy{}\PYZgt{}S \PY{l+m}{20}
C\PYZhy{}\PYZgt{}S \PY{l+m}{30}
C\PYZhy{}\PYZgt{}S
S\PYZhy{}\PYZgt{}C \PY{l+m}{100} OK
S\PYZhy{}\PYZgt{}C
C\PYZhy{}\PYZgt{}S MULTIPLY
C\PYZhy{}\PYZgt{}S
S\PYZhy{}\PYZgt{}C \PY{l+m}{100} OK
S\PYZhy{}\PYZgt{}C
C\PYZhy{}\PYZgt{}S PEEK
C\PYZhy{}\PYZgt{}S
S\PYZhy{}\PYZgt{}C \PY{l+m}{100} OK
S\PYZhy{}\PYZgt{}C \PY{l+m}{600}
S\PYZhy{}\PYZgt{}C
C\PYZhy{}\PYZgt{}S ADD
C\PYZhy{}\PYZgt{}S
S\PYZhy{}\PYZgt{}C \PY{l+m}{100} OK
S\PYZhy{}\PYZgt{}C
C\PYZhy{}\PYZgt{}S PEEK
C\PYZhy{}\PYZgt{}S
S\PYZhy{}\PYZgt{}C \PY{l+m}{100} OK
S\PYZhy{}\PYZgt{}C \PY{l+m}{610}
S\PYZhy{}\PYZgt{}C
\end{Verbatim}

\end{framed}

\end{document}
